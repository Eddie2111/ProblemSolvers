\documentclass{article}
\usepackage[utf8]{inputenc}
\begin{document}
\title{Assignment 2}
\author{\\A S M TAREQ MAHMOOD\\ID:20101073\\SET-14\\}
\date{March 2021}
\maketitle
\pagebreak


\section{Task-1}
Given that,
\[ \int_{1}^{exp(\pi/8)} \frac{sin^3(8ln(x))cos^7}{x} \,dx \]
let,
$4ln(x)=z\\$
$4 \frac{1}{x}= \frac{dz}{dx}\\$
$ dx = \frac{dz}{4}x$
\\if,
$x =1 ; z=0$
\\if,
$ x = e^{\frac{\pi}{8}}\\$
So,
$\\z =4 \frac{\pi}{8}$
$\\z = \frac{\pi}{2}$
\[=>\int_{1}^{exp{\frac{\pi}{8}}} \frac{sin^3 (8ln(x)) cos^7(4ln(x))}{x} dx \]
\[=>\int_{0}^{\frac{\pi}{2}} \frac{sin^3(2z) cos^7(z)}{x} \frac{dz}{4} x \]
\[=> \frac{1}{4}  \int_{0}^{\frac{\pi}{0}} (2sinzcosz)^3 cos^7(z) dz\]
\[=> \frac{1}{4} 8 \int_{0}^{\frac{\pi}{2}} sin^3(z)cos^{10}(z) dz\]
\[=> 2 \int_{0}^{\frac{\pi}{2}} sin^3(z)cos^{10}(z) dz\]
Now,
$2m-1=3\\$
$m=2$
And,
$\\2n-1=10$	
$\\n=\frac{11}{2}$
\\We Know,
\[\beta (x,y) = \int_{0}^{\frac{\pi}{2}} 2sin^{2x-1}(t)cos^{2y-1}(t) dt\]
\[\beta (2,\frac{11}{2}) = \int_{0}^{\frac{\pi}{2}} 2sin^{3}(t)cos^{10}(t) dt\]
\[=> \frac{\gamma (2) \gamma(\frac{11}{2})}{\gamma (2+ \frac{11}{2})}          \]
\[=> \frac{\gamma (2) \gamma(\frac{11}{2})}{\gamma (\frac{15}{11})}          \]
\[=> \frac{\sqrt{\pi} (\frac{945 \sqrt{\pi}}{32})} {(\frac{135135 \sqrt{\pi}}{128})} \]
\[=> \frac{4 \sqrt{\pi}}{143}\]
\newpage

\section{Task-2}
Given that,
\[=>\int{\frac{cos^3(x)}{sin(x)}dx} \]
\[=>\int \frac{cos^3 (x)-3cos(x)sin^2x}{sin(x)}dx\]
\[=>-\int \frac{3cos(x)sin(x)-cos^3(x)}{sin(x)}dx\]
\[=>-\int \frac{(4sin^2-1) cos(x)}{sin(x)}dx\]
Let,
\\u = sin(x)\\
\\ du/dx = cos(x)\\
\\ dx = {1/cos(x)}\,dx\\
\\So,
\[=>\int \frac{4u^2-1}{u}\,du\]
\[=>\int (4u-\frac{1}{u})\,du\]
\[=>4\int(u)\,du - \int \frac{1}{u}\,du ====(i)\]
\\Again,
$\int(u\,du)\\$
$ \frac{u^2}{2}=====(ii)\\$
\\And,
\\$\\ \frac{1}{u}\,du\\$
$ ln(u)=====(iii)$\\
\\From ii and iii
\[=>2u^2-ln(u)\] 
\[=>2sin^2x-ln(sin(x))\]
\\We get,
\[=>-\int \frac{3cos(x)sin^2(x)-cos^3(x)}{sin(x)}\]
\[=>lnsin(x)-2sin^2(x)\]

\newpage

\section{Task-3}
Given that,
\[\int cos(sec\theta)sec\theta tan\theta d\theta\]
\\Let,\\
$u = sec\theta$\\ \\
$\frac{du}{d\theta}= sec\theta tan\theta$\\  \\
$d\theta = \frac{1}{sec\theta tan\theta}$ \\ \\ \\
After Implementing,
\[=>\int cos(\frac{sec\theta}{sec\theta tan\theta}) sec\theta tan\theta d\theta\]
\[=>\int cos(sec\theta)d\theta\]
\[=>sin(sec\theta)+c\]

\end{document}
